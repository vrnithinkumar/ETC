\documentclass[11pt]{article}

\usepackage{mathpartir}

% -- Semantics stuff
%     I defined a couple of special commands (see examples in the text below) to make
%     writing inference rules and judgements easier.
\newcommand{\br}[1]{\langle #1 \rangle}
\def\Yields{\Downarrow}

% -- Page size
\textheight     9.0truein
\textwidth      6.5truein
\topmargin     -0.5truein
\oddsidemargin  +0.0truein
\evensidemargin +0.0truein

% -- Document title (appears at top)
\title{COMP 105 Homework}
\author{Samuel Z. Guyer}

\begin{document}

\maketitle

% -- Example of a "heading": a bold title, not indented
%     Any text inside the \textbf{} markup is rendered in bold
\noindent
\textbf{Exercise 1}

% -- Regular paragraph

Regular text paragraphs need  no special markup. Paragraphs are indented and word-wrapped automatically. 
White space is not significant, but paragraphs are separated by any number of blank lines.

% -- Formatting code
%     The verbatim mode renders text in typewriter font and preserves every character, 
%     including white space, exactly as you write it.
\begin{verbatim}
    int main(int argc, char * argv[])
    {
        printf("Hello, world\n");
        return 0;
    }
\end{verbatim}

% -- Mathematics
%     Mathematical formulas are rendered in a special "math mode". For small formulas
%     that you want embedded in other text, you enclose the formula in dollar signs. For
%     larger formulas, like operational semantics inference rules, you can use the mathpar
%     environment.  The \br command is defined in this document (at the top)

An operational semantics judgement, like ${\br{e_1, \xi, \phi, \rho} \Yields \br{v_1, \xi, \phi, \rho'}}$, can be 
embedded in a paragraph using dollar signs.

To write a whole inference rule, use the mathpar markup. The inferrule markup has three parts:

\begin{verbatim}
\infferrule*[Right=formulaname]
  { premises }
  { conclusion }
\end{verbatim}

Here is an example:

\begin{mathpar}
\inferrule*[Right=\textsc{IfTrue}]
  {\br{e_1, \xi, \phi, \rho} \Yields \br{v_1, \xi, \phi, \rho'}\\\\
    v_1 \ne 0\\\\
    \br{e_2, \xi, \phi, \rho'} \Yields \br{v_2, \xi, \phi, \rho''}\\ }
  {\br{IF(e_1, e_2, e_3), \xi, \phi, \rho} \Yields \br{v_1, \xi, \phi, \rho''}}
\end{mathpar}

\end{document}