
\documentclass{amsart}
\usepackage{amsmath, amssymb}
\usepackage{mathpartir, proof}
\title{Erlang - Bidirectional Typing}

\newcommand{\type}{\ensuremath{\mathtt{type}}}
\newcommand{\application}{\Rightarrow\!\!\!\Rightarrow }
\begin{document}
\maketitle

% \begin{abstract}
% \end{abstract}

\begin{section}{Eralng Syntax}
\[
\begin{array}{lcl}
  fname & ::= & Atom / Integer \\
  const & ::= & lit \mid [ const_1 \mid const_2 ] \mid \{ const_1, . . ., const_n \} \\
  lit & ::= & Integer \mid Float \mid Atom \mid Char \mid String \mid [\ ] \\
  fun & ::= & \textbf{fun} (var_1,..., var_n) \to exprs \\
  var & ::= & VariableName \\
  exprs &  ::= & expr \mid\ < expr_1,..., expr_n > \\
  expr & ::=  & var \mid fname \mid lit \mid \textbf{fun} \\ 
  & \mid & [ exprs_1 \mid exprs_2 ] \mid \{ exprs_1,..., exprs_n \} \\ 
  & \mid & \textbf{let}\ vars = exprs_1\ \textbf{in}\ exprs_2 \\ 
  & \mid & \textbf{case}\ exprs\ \textbf{of}\ clause_1\ ... clause_n\ \textbf{end} \\ 
  & \mid & \textbf{letrec}\ fname_1\ = fun_1 ... fname_n = fun_n \textbf{in}\ exprs \\ 
  & \mid & \textbf{apply}\ exprs_0(exprs_1,..., exprsn) \\ 
  & \mid & \textbf{call}\ exprs_0\ 1:exprs_0 2(exprs_1,..., exprs_n) \\ 
  & \mid & \textbf{primop}\ Atom(exprs_1,..., exprs_n) \\ 
  & \mid & \textbf{receive}\ clause_1 ... clause_n\ after\ exprs_1 \to exprs_2 \\ 
  & \mid & \textbf{try}\ exprs_1\ of\ <var_1, ...var_n> \to exprs_2\ \textbf{catch}\ <var_{n+1},...,var_{n+m}> \to exprs_3 \\ 
  & \mid & \textbf{do}\ exprs_1\ exprs_2 \\ 
  & \mid & \textbf{catch}\ exprs \\
  vars & ::= & var \mid\ <var_1, . . ., var_n>\\
  clause & ::= & pats\ \textbf{when}\ exprs_1 \to exprs_2\\
  pats & ::= & pat \mid\ <pat_1, . . ., pat_n>\\
  pat & ::= & var \mid lit \mid [ pat_1 \mid pat_2 ] \mid \{pat_1, . . ., pat_n\} \mid var = pat \\
\end{array}
\]
\end{section}

\begin{section}{Types and Syntax}
\[
\begin{array}{lcl}
  Type & ::= & any()\\           
  & \mid &  none()  \\              
  & \mid & pid()\\
  & \mid & port()\\
  & \mid & reference()\\
  & \mid & [\ ]       \\             
  & \mid & Atom\\
  & \mid & Bitstring\\
  & \mid & float()\\
  & \mid & Fun\\
  & \mid & Integer\\
  & \mid & List\\
  & \mid & Map\\
  & \mid & Tuple\\
  & \mid & Union\\
  & \mid & UserDefined \\
\\
  Atom & ::= & atom()\\
  & \mid & Erlang\_Atom \\

% Bitstring :: <<>>\\
%              | <<_:M>>\\
%              | <<_:_*N>>\\
%              | <<_:M, _:_*N>>\\
\\
  Fun & ::= & fun()\\
  & \mid & fun((...) \to Type)\\
  & \mid & fun(() \to Type)\\
  & \mid & fun((TList) \to Type)\\

%   Integer :: integer()\\
%           | Integer_Value\\
%           | Integer_Value..Integer_Value\\

%   Integer_Value :: Erlang_Integer              %% ..., -1, 0, 1, ... 42 ...
%                  | Erlang_Character            %% $a, $b ...
%                  | Integer_Value BinaryOp Integer_Value
%                  | UnaryOp Integer_Value

%   BinaryOp :: '*' | 'div' | 'rem' | 'band' | '+' | '-' | 'bor' | 'bxor' | 'bsl' | 'bsr'

%   UnaryOp :: '+' | '-' | 'bnot'

%   List :: list(Type)                           %% Proper list ([]-terminated)
%         | maybe_improper_list(Type1, Type2)    %% Type1=contents, Type2=termination
%         | nonempty_improper_list(Type1, Type2) %% Type1 and Type2 as above
%         | nonempty_list(Type)                  %% Proper non-empty list
\\
  Map & ::= & \#\{\}       \\
  & \mid & \#\{AssociationList\} \\
\\
  Tuple & ::= & tuple() \\
  & \mid & \{\} \\ 
  & \mid & \{TList\} \\

%   AssociationList :: Association
%                   | Association, AssociationList

%   Association :: Type := Type                  %% denotes a mandatory association
%               | Type => Type                  %% denotes an optional association

%   TList :: Type
%          | Type, TList
\\
  Union & ::=  & Type_1 \mid Type_2
\end{array}
\]  
\end{section}  
\newpage
\begin{section}{Type Rules}
\begin{subsection}{Constants}
\begin{mathpar}
  \inferrule[Const]{\quad}{\Gamma \vdash C \Rightarrow \type_c}\and
\end{mathpar}
Example:\\
\begin{mathpar}
  \inferrule[Int]{\quad}{\Gamma \vdash 0 \Rightarrow integer()}\and
    \inferrule[Bool]{\quad}{\Gamma \vdash true \Rightarrow boolean()}\and
\end{mathpar}
\end{subsection}

\begin{subsection}{Variables}
\begin{mathpar}
  \inferrule[Var]{(x : A) \in \Gamma}{\Gamma \vdash x \Rightarrow A}\and
\end{mathpar}
\end{subsection}

\begin{subsection}{Annotation}
\begin{mathpar}
  \inferrule[Ann]{\Gamma \vdash x \Leftarrow A}{\Gamma \vdash (x: A) \Rightarrow A}\and
\end{mathpar}
\end{subsection}

\begin{subsection}{Case}
\begin{mathpar}
  \inferrule[Case-Inf]{\Gamma \vdash e \Rightarrow (A_1 \mid A_2) \\
            \Gamma, x_1 : A_1 \vdash e_1 \Rightarrow B_1 \\
            \Gamma, x_2 : A_2 \vdash e_2 \Rightarrow B_2}
            {\Gamma \vdash case(e,\ inj_1\ x_1.e_1,\ inj_2\ x_2.e_2) \Rightarrow (B_1 \mid B_2)} \and
  \inferrule[Case-Check]{\Gamma \vdash e \Rightarrow (A_1 \mid A_2) \\
            \Gamma, x_1 : A_1 \vdash e_1 \Leftarrow B_1 \\
            \Gamma, x_2 : A_2 \vdash e_2 \Leftarrow B_2}
            {\Gamma \vdash case(e,\ inj_1\ x_1.e_1,\ inj_2\ x_2.e_2) \Leftarrow (B_1 \mid B_2)}
\end{mathpar}
\end{subsection}

\begin{subsection}{Clause}
\begin{mathpar}
  \inferrule[Clause-Check]{\Gamma, pat : A \vdash e \Leftarrow B}
  {\Gamma \vdash clause(pat,\ e) \Leftarrow A \to B}\and

  \inferrule[Clause-Inf*]{\Gamma \vdash pat \Rightarrow A \\
  \Gamma, pat : A \vdash e \Rightarrow B}
  {\Gamma \vdash clause(pat,\ e) \Rightarrow A \to B}\and
\end{mathpar}
\\
\textit{*May have to use type variable and local constraint solver.}
\end{subsection}

\begin{subsection}{Function}
\begin{mathpar}
  \inferrule[Function-Check]{\Gamma \vdash x \Rightarrow (A_1 \mid A_2) \\
            \Gamma, x : A_1 \vdash clause_1 \Leftarrow (A_1 \mid A_2) \to (B_1 \mid B_2) \\
            \Gamma, x : A_2 \vdash clause_2 \Leftarrow (A_1 \mid A_2) \to (B_1 \mid B_2) }
  {\Gamma \vdash fun(x, clause_1, clause_2) \Leftarrow (A_1 \mid A_2) \to (B_1 \mid B_2)}\and
\end{mathpar}
\begin{mathpar}
  \inferrule[Function-Inf]{\Gamma \vdash x \Rightarrow (A_1 \mid A_2) \\
            \Gamma, x : A_1 \vdash clause_1 \Rightarrow A_1 \to B_1 \\
            \Gamma, x : A_2 \vdash clause_2 \Rightarrow A_2 \to B_2 }
  {\Gamma \vdash fun(x, clause_1, clause_2) \Rightarrow (A_1 \to B_1) \mid (A_2 \to B_2)}\and
\end{mathpar}
\end{subsection}

\begin{subsection}{Let}
\begin{mathpar}
  \inferrule[Let-Infer]{\Gamma \vdash e_1 \Rightarrow A\\
  \Gamma, x : A \vdash e_2 \Rightarrow B}
  {\Gamma \vdash let\ x = e_1\ in\ e_2 \Rightarrow B}\and
\end{mathpar}
\begin{mathpar}
  \inferrule[Let-Check]{\Gamma \vdash e_1 \Rightarrow A\\
  \Gamma, x : A \vdash e_2 \Leftarrow B}
  {\Gamma \vdash let\ x = e_1\ in\ e_2 \Leftarrow B}\and
\end{mathpar}
\begin{mathpar}
  \inferrule[Let-Check-Erl]{\Gamma \vdash e_1 \Rightarrow A\\
  \Gamma, x : A \vdash e_2 \Leftarrow B}
  {\Gamma \vdash x = e_1, e_2 \Leftarrow B}\and
\end{mathpar}
\end{subsection}

\begin{subsection}{If-Condition}
\begin{mathpar}
  \inferrule[IF-Cond]{\Gamma \vdash x_1 \Leftarrow Bool \\
  \Gamma \vdash x_2 \Leftarrow Bool \\
  \Gamma \vdash e_1 \Rightarrow B_1 \\
  \Gamma \vdash e_2 \Rightarrow B_2
  }
  {\Gamma \vdash if(x_1 \to e_1, x_2 \to e_2) \Rightarrow (B_1 \mid B_2)}\and
\end{mathpar}
\end{subsection}

\begin{subsection}{Subtyping}
\begin{mathpar}
  \inferrule[Sub]{\Gamma \vdash e \Rightarrow A \\
  A <: B
  }
  {\Gamma \vdash e \Leftarrow B }
\end{mathpar}
\end{subsection}

\begin{subsection}{Record}
\end{subsection}

\end{section}
\newpage
TODO: INSERT Algorithmic typing
\newpage

\begin{section}{Higher Rank Polymorphism}
\begin{subsection}{Poly-Identity Application}

(id: $\forall \alpha.\ \alpha \to \alpha$)

\begin{mathpar}
\inferrule*[Right=\textsc{$\forall$App}]
{  
    \inferrule*[Right=\textsc{$\to$App}]
    {
        \inferrule*[Right=\textsc{Sub}]
        {   
            \Gamma, \tau \vdash e \Rightarrow \tau \dashv \Theta \\
            \inferrule*[Right=\textsc{<:InstantiateR}]
            {   
                %\hat{\alpha} \notin FV() \\
                \inferrule*[Right=\textsc{<:InstaRSolve}]
                {   
                    \Gamma \vdash \tau
                } 
                {   
                    \Gamma, \hat{\alpha} \vdash \hat{\alpha}  :\leqq \tau \dashv \Delta, \hat{\alpha} = \tau
                }
            } 
            {   
                \Theta \vdash \tau <: \hat{\alpha} \dashv \Delta, \hat{\alpha} = \tau
            }
        } 
        {
            \Gamma, \hat{\alpha} \vdash e \Leftarrow \hat{\alpha}  \dashv \Delta, \hat{\alpha} = \tau
        }
    }
    {
        \Gamma, \hat{\alpha} \vdash (\hat{\alpha} \to \hat{\alpha} )\ e \application \hat{\alpha}  \dashv \Delta, \hat{\alpha} = \tau
    }
}
{
\Gamma \vdash (\forall \alpha.\ \alpha \to \alpha)\ e \application \tau \dashv \Delta
}
\end{mathpar}
\end{subsection}

\begin{subsection}{Poly-Op}
\begin{mathpar}
  \inferrule*[Right=\textsc{$\forall$App}]
  {
    \inferrule
    {
        \Gamma, \hat{\alpha}, e_1:\hat{\alpha} \vdash e_1 \Rightarrow \tau  \dashv \Theta, \hat{\alpha} = \tau \\
        \Theta , e_2:\hat{\alpha} \vdash e_2 \Leftarrow \tau \dashv \Delta, \hat{\alpha} = \tau
    }
    {
        \Gamma, \hat{\alpha} \vdash ((\hat{\alpha} \to \hat{\alpha} \to \hat{\alpha}) e_1) e_2) \application \tau \dashv \Delta
    }
  }
  {
    \Gamma \vdash ((\forall \alpha.\ \alpha \to \alpha \to \alpha) e_1) e_2) \application \tau \dashv \Delta
  }
\end{mathpar}
Op: $\forall \alpha.\ \alpha \to \alpha \to \alpha$
\end{subsection}

\end{section}
\end{document}
%   \inferrule[1b]{\Gamma \vdash \tau_1 : \type \\
%              \Gamma \vdash \tau_1 : \type}
%             {\Gamma \vdash \tau_1 \to \tau_2 : \type}\and
%   \inferrule[1c]{\Gamma, \alpha : \type \vdash \tau : \type}
%             {\Gamma \vdash \forall \alpha. \tau : \type}\and
%   \inferrule[1d]{x : \tau \in \Gamma}{\Gamma \vdash x : \tau}\and
%   \inferrule{\Gamma, x : \tau_1 \vdash e : \tau_2 \\
%              \Gamma \vdash \tau : \type}
%             {\Gamma \vdash \lambda x : \tau_1. e : \tau_1 \to \tau_2}\and
%   \inferrule[1e]{\Gamma \vdash e_1 : \tau_1 \to \tau_2 \\
%              \Gamma \vdash e_2 : \tau_2}
%             {\Gamma \vdash e_1\ e_2 : \tau_2}\and
%   \inferrule[1f]{\Gamma, \alpha : \type \vdash e : \tau}
%             {\Gamma \vdash \Lambda \alpha. e : \forall \alpha. \tau}\and
%   \inferrule[1g]{\Gamma \vdash e : \forall \alpha. \tau_2 \\
%              \Gamma \vdash \tau_1 : \type}
%             {\Gamma \vdash e[\tau_1] : [\tau_1/\alpha]\tau_2}\and